\documentclass{article}
\usepackage{amsmath, amssymb, amsthm}
\usepackage{tikz}
\usetikzlibrary{graphs,graphs.standard}

\begin{document}

\title{Matrices de Adyacencia e Incidencia}
\date{6 de septiembre de 2023}
\maketitle

\section*{Gráfica \( G_1 \)}


Justificación:
\begin{itemize}
	\item \( G_1 \) tiene 10 vértices.
	\item Los grados de los vértices de \( G_1 \) son: 3, 3, 3, 4, 2, 3, 3, 2, 3, 2.
\end{itemize}

\textbf{Matriz de Adyacencia \( G_1 \)}:
\[
A_1 =
\begin{array}{c|cccccccccc}
    & v1 & v2 & v3 & v4 & v5 & v6 & v7 & v8 & v9 & v10 \\
\hline
v1 & 0 & 1 & 0 & 0 & 1 & 1 & 0 & 0 & 0 & 0 \\
v2 & 1 & 0 & 1 & 0 & 0 & 0 & 1 & 0 & 0 & 0 \\
v3 & 0 & 1 & 0 & 1 & 0 & 0 & 0 & 1 & 0 & 0 \\
v4 & 0 & 0 & 1 & 0 & 1 & 0 & 0 & 0 & 1 & 1 \\
v5 & 1 & 0 & 0 & 1 & 0 & 0 & 0 & 0 & 0 & 0 \\
v6 & 1 & 0 & 0 & 0 & 0 & 0 & 0 & 1 & 1 & 0 \\
v7 & 0 & 1 & 0 & 0 & 0 & 0 & 0 & 0 & 1 & 1 \\
v8 & 0 & 0 & 1 & 0 & 0 & 1 & 0 & 0 & 0 & 0 \\
v9 & 0 & 0 & 0 & 1 & 0 & 1 & 1 & 0 & 0 & 0 \\
v10& 0 & 0 & 0 & 1 & 0 & 0 & 1 & 0 & 0 & 0 \\
\end{array}
\]


\textbf{Matriz de Incidencia \( G_1 \)}:

\setcounter{MaxMatrixCols}{20}
\[
M_1 =
\begin{array}{c|cccccccccccccc}
    & e1 & e2 & e3 & e4 & e5 & e6 & e7 & e8 & e9 & e10 & e11 & e12 & e13 & e14 \\
\hline
v1 & 1  & 1  & 1  & 0  & 0  & 0  & 0  & 0  & 0  & 0  & 0  & 0  & 0  & 0   \\
v2 & -1 & 0  & 0  & 1  & 1  & 0  & 0  & 0  & 0  & 0  & 0  & 0  & 0  & 0   \\
v3 & 0  & 0  & 0  & -1 & 0  & 1  & 1  & 0  & 0  & 0  & 0  & 0  & 0  & 0   \\
v4 & 0  & 0  & 0  & 0  & 0  & -1 & 0  & 1  & 1  & 1  & 0  & 0  & 0  & 0   \\
v5 & 0  & -1 & 0  & 0  & 0  & 0  & 0  & -1 & 0  & 0  & 0  & 0  & 0  & 0   \\
v6 & 0  & 0  & -1 & 0  & 0  & 0  & 0  & 0  & 0  & 0  & 1  & 1  & 0  & 0   \\
v7 & 0  & 0  & 0  & 0  & -1 & 0  & 0  & 0  & 0  & 0  & 0  & 0  & 1  & 1   \\
v8 & 0  & 0  & 0  & 0  & 0  & 0  & -1 & 0  & 0  & 0  & -1 & 0  & 0  & 0   \\
v9 & 0  & 0  & 0  & 0  & 0  & 0  & 0  & 0  & -1 & 0  & 0  & -1 & -1 & 0   \\
v10& 0  & 0  & 0  & 0  & 0  & 0  & 0  & 0  & 0  & -1 & 0  & 0  & 0  & -1  \\
\end{array}
\]

\section*{Gráfica \( G_2 \)}


Justificación:
\begin{itemize}
    \item \( G_2 \) tiene 10 vértices.
    \item Los grados de los vértices de \( G_2 \) son: 3, 3, 3, 4, 3, 3, 3, 3, 3, 3.
    \item \( G_1 \) no es isomorfa a \( G_2 \) ya que, aunque tienen el mismo número de vértices y aristas, sus conjuntos de grados de vértices son diferentes.
\end{itemize}


\textbf{Matriz de Adyacencia \( G_2 \)}:
\[
A_2 =
\begin{array}{c|cccccccccc}
    & v1 & v2 & v3 & v4 & v5 & v6 & v7 & v8 & v9 & v10 \\
\hline
v1 & 0 & 1 & 0 & 0 & 0 & 0 & 0 & 0 & 1 & 1 \\
v2 & 1 & 0 & 1 & 0 & 0 & 1 & 0 & 0 & 0 & 0 \\
v3 & 0 & 1 & 0 & 1 & 0 & 0 & 0 & 1 & 0 & 0 \\
v4 & 0 & 0 & 1 & 0 & 1 & 0 & 0 & 0 & 0 & 1 \\
v5 & 0 & 0 & 0 & 1 & 0 & 1 & 0 & 0 & 1 & 0 \\
v6 & 0 & 1 & 0 & 0 & 1 & 0 & 1 & 0 & 0 & 0 \\
v7 & 0 & 0 & 0 & 0 & 0 & 1 & 0 & 1 & 0 & 1 \\
v8 & 0 & 0 & 1 & 0 & 0 & 0 & 1 & 0 & 1 & 0 \\
v9 & 1 & 0 & 0 & 0 & 1 & 0 & 0 & 1 & 0 & 0 \\
v10& 1 & 0 & 0 & 1 & 0 & 0 & 1 & 0 & 0 & 0 \\
\end{array}
\]

\textbf{Matriz de Incidencia \( G_2 \)}:

\setcounter{MaxMatrixCols}{20}
\[
M_2 =
\begin{array}{c|ccccccccccccccc}
    & e1 & e2 & e3 & e4 & e5 & e6 & e7 & e8 & e9 & e10 & e11 & e12 & e13 & e14 & e15 \\
\hline
v1 & 1  & 1  & 1  & 0  & 0  & 0  & 0  & 0  & 0  & 0  & 0  & 0  & 0  & 0  & 0   \\
v2 & -1 & 0  & 0  & 1  & 1  & 0  & 0  & 0  & 0  & 0  & 0  & 0  & 0  & 0  & 0   \\
v3 & 0  & 0  & 0  & -1 & 0  & 1  & 1  & 0  & 0  & 0  & 0  & 0  & 0  & 0  & 0   \\
v4 & 0  & 0  & 0  & 0  & 0  & -1 & 0  & 1  & 1  & 0  & 0  & 0  & 0  & 0  & 0   \\
v5 & 0  & 0  & 0  & 0  & 0  & 0  & 0  & -1 & 0  & 1  & 1  & 0  & 0  & 0  & 0   \\
v6 & 0  & 0  & 0  & 0  & -1 & 0  & 0  & 0  & 0  & -1 & 0  & 1  & 0  & 0  & 0   \\
v7 & 0  & 0  & 0  & 0  & 0  & 0  & 0  & 0  & 0  & 0  & 0  & -1 & 1  & 1  & 0   \\
v8 & 0  & 0  & 0  & 0  & 0  & 0  & -1 & 0  & 0  & 0  & 0  & 0  & -1 & 0  & 1   \\
v9 & 0  & -1 & 0  & 0  & 0  & 0  & 0  & 0  & 0  & 0  & -1 & 0  & 0  & 0  & -1  \\
v10& 0  & 0  & -1 & 0  & 0  & 0  & 0  & 0  & -1 & 0  & 0  & 0  & 0  & -1 & 0   \\
\end{array}
\]

\section*{Gráfica \( G_3 \)}



Justificación:
\begin{itemize}
    \item \( G_3 \) tiene 6 vértices.
    \item Los grados de los vértices de \( G_3 \) son: 3, 2, 2, 4, 3, 2.
\end{itemize}


\textbf{Matriz de Adyacencia \( G_3 \)}:
\[
A_3 =
\begin{array}{c|cccccc}
    & x1 & x2 & x3 & x4 & x5 & x6 \\
\hline
x1 & 0 & 1 & 1 & 1 & 0 & 0 \\
x2 & 1 & 0 & 0 & 1 & 0 & 0 \\
x3 & 1 & 0 & 0 & 0 & 1 & 0 \\
x4 & 1 & 1 & 0 & 0 & 1 & 1 \\
x5 & 0 & 0 & 1 & 1 & 0 & 1 \\
x6 & 0 & 0 & 0 & 1 & 1 & 0 \\
\end{array}
\]


\textbf{Matriz de Incidencia \( G_3 \)}:
\[
M_3 =
\begin{array}{c|cccccccc}
    & e1 & e2 & e3 & e4 & e5 & e6 & e7 & e8 \\
\hline
x1 & 1  & 1  & 1  & 0  & 0  & 0  & 0  & 0  \\
x2 & -1 & 0  & 0  & 1  & 0  & 0  & 0  & 0  \\
x3 & 0  & -1 & 0  & 0  & 1  & 0  & 0  & 0  \\
x4 & 0  & 0  & -1 & -1 & 0  & 1  & 1  & 0  \\
x5 & 0  & 0  & 0  & 0  & -1 & -1 & 0  & 1  \\
x6 & 0  & 0  & 0  & 0  & 0  & 0  & -1 & -1 \\
\end{array}
\]


\section*{Gráfica \( G_4 \)}


Justificación:
\begin{itemize}
    \item \( G_4 \) tiene 6 vértices.
    \item Los grados de los vértices de \( G_4 \) son: 3, 2, 3, 4, 2, 2.
    \item \( G_3 \) es isomorfa a \( G_4 \) ya que tienen el mismo número de vértices, el mismo número de aristas y el mismo conjunto de grados de vértices. Basta con mover la arista \( x_4x_5 \) a \( x_3x_4 \) para obtener \( G_4 \).
\end{itemize}




\textbf{Matriz de Adyacencia \( G_4 \)}:
\[
A_4 =
\begin{array}{c|cccccc}
    & w1 & w2 & w3 & w4 & w5 & w6 \\
\hline
w1 & 0 & 1 & 1 & 1 & 0 & 0 \\
w2 & 1 & 0 & 0 & 1 & 0 & 0 \\
w3 & 1 & 0 & 0 & 1 & 1 & 0 \\
w4 & 1 & 1 & 1 & 0 & 0 & 1 \\
w5 & 0 & 0 & 1 & 0 & 0 & 1 \\
w6 & 0 & 0 & 0 & 1 & 1 & 0 \\
\end{array}
\]



\textbf{Matriz de Incidencia \( G_4 \)}:
\[
M_4 =
\begin{array}{c|cccccccc}
    & e1 & e2 & e3 & e4 & e5 & e6 & e7 & e8 \\
\hline
w1 & 1  & 1  & 1  & 0  & 0  & 0  & 0  & 0  \\
w2 & -1 & 0  & 0  & 1  & 0  & 0  & 0  & 0  \\
w3 & 0  & -1 & 0  & 0  & 1  & 1  & 0  & 0  \\
w4 & 0  & 0  & -1 & -1 & -1 & 0  & 1  & 0  \\
w5 & 0  & 0  & 0  & 0  & 0  & -1 & 0  & 1  \\
w6 & 0  & 0  & 0  & 0  & 0  & 0  & -1 & -1 \\
\end{array}
\]




\end{document}
