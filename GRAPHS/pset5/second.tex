\documentclass{article}
\usepackage{amsmath, amssymb, amsthm}

\begin{document}

\section*{Problema}
3.Sea \( G \) una gráfica con exactamente 2 vértices de grado impar, denominados \( u \) y \( v \). Demuestra que \( G \) tiene al menos una uv-trayectoria.

\begin{Dem.}
Si \( G \) es conexa, entonces \( G \) tiene una uv-trayectoria. Ahora el caso donde \( G \) es desconexa.

Sea \( K \) la componente conexa de \( G \) tal que \( u \in K \). Dado que \( K \) es una gráfica, debe contener un número par de vértices de grado impar ya que
\begin{equation}
\sum_{w \in V(K)} \text{grado}(w) = 2|E(K)|.
\end{equation}

Esto implica que hay al menos un vértice más de grado impar en \( K \). Dado que \( G \) contiene solo dos vértices de grado impar, debe ser el caso de que \( v \in K \), lo que indica que existe una uv-trayectoria en \( K \).

Por lo tanto, en ambos casos, tanto si \( G \) es conexa como desconexa, se tiene que \( G \) contiene al menos una uv-trayectoria.

\end{Dem.}

\end{document}
