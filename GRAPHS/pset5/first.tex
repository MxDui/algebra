\documentclass{article}
\usepackage{amsmath, amssymb, amsthm}

\begin{document}

\section*{Problema}
2.Sea \(G\) una gráfica con 19 vértices y 4 componentes conexas. Demuestra que al menos una componente conexa tiene 5 o más vértices.

\begin{Dem.}

Supongamos, por contradicción, que todas las componentes conexas de \(G\) tienen a lo más 4 vértices.

Si tomamos el máximo número posible de vértices por componente conexa, tendríamos: \(4 \times 4 = 16\) vértices en total {\Rightarrow\!\Leftarrow}


Esto es la contradicción porque \(G\) tiene 19 vértices.

Por lo tanto, como \(G\) tiene 19 vértices y con 4 componentes conexas de a lo más 4 vértices sólo se obtienen 16 vértices, necesariamente al menos una de las componentes conexas de \(G\) debe tener 5 o más vértices.

\end{Dem.}

\end{document}