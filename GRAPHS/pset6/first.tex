\documentclass{article}

\begin{document}

\title{Resolución del problema de las carreteras}
\author{David Rivera Morales}
\maketitle

\section{Planteamiento del problema}
Una red de carreteras pasa por 35 ciudades. Si existe una única carretera entre cualesquiera dos ciudades, y además se sabe que:
\begin{itemize}
    \item 25 son terminales de las carreteras.
    \item En 2 solo llegan dos carreteras.
    \item En 3 llegan 4 carreteras.
    \item En 1 llega 5 carreteras.
    \item En 2 llegan 6 carreteras.
    \item La red tiene dos ciudades donde llegan $x$ carreteras.
\end{itemize}
¿Cuál es el valor de $x$?

\section{Solución}
Para resolver el problema, consideramos el número total de aristas basado en la información proporcionada:
\begin{itemize}
    \item 25 ciudades contribuyen con $25 \times 1 = 25$ al grado total.
    \item 2 ciudades contribuyen con $2 \times 2 = 4$ al grado total.
    \item 3 ciudades contribuyen con $3 \times 4 = 12$ al grado total.
    \item 1 ciudad contribuye con $1 \times 5 = 5$ al grado total.
    \item 2 ciudades contribuyen con $2 \times 6 = 12$ al grado total.
    \item 2 ciudades contribuyen con $2x$ al grado total.
\end{itemize}

La suma total de estos grados debe ser igual al doble del número de aristas. Como hay una carretera única entre cualesquiera dos ciudades y hay 35 ciudades, el número total de carreteras es $\frac{35 \times 34}{2} = 595$. 

Por lo tanto, planteamos la ecuación:
\[
25 + 4 + 12 + 5 + 12 + 2x = 2 \times 595
\]

Resolviendo esta ecuación, encontramos que:
\[
x = 566
\]

Por lo tanto, hay dos ciudades en la red donde llegan 566 carreteras a cada una.

\end{document}
