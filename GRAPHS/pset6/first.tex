\documentclass{article}

\begin{document}


\section{Planteamiento del problema}
Una red de carreteras pasa por 35 ciudades. Si existe una única carretera entre cualesquiera dos ciudades, y además se sabe que:
\begin{itemize}
    \item 25 son terminales de las carreteras.
    \item En 2 solo llegan dos carreteras.
    \item En 3 llegan 4 carreteras.
    \item En 1 llega 5 carreteras.
    \item En 2 llegan 6 carreteras.
    \item La red tiene dos ciudades donde llegan $x$ carreteras.
\end{itemize}
¿Cuál es el valor de $x$?

\section{Solución basada en la teoría de árboles}
Para resolver el problema, consideraremos la red de carreteras como un árbol. En un árbol, la suma de los grados de todos los nodos es igual al doble del número de aristas. Dado que hay 35 ciudades (nodos), hay 34 carreteras (aristas). 

Vamos a calcular el grado total basado en la información proporcionada:
\begin{itemize}
    \item 25 ciudades contribuyen con $25 \times 1 = 25$ al grado total.
    \item 2 ciudades contribuyen con $2 \times 2 = 4$ al grado total.
    \item 3 ciudades contribuyen con $3 \times 4 = 12$ al grado total.
    \item 1 ciudad contribuye con $1 \times 5 = 5$ al grado total.
    \item 2 ciudades contribuyen con $2 \times 6 = 12$ al grado total.
    \item 2 ciudades contribuyen con $2x$ al grado total.
\end{itemize}

Por lo tanto, planteamos la ecuación:
\[
25 + 4 + 12 + 5 + 12 + 2x = 2 \times 34
\]

Resolviendo esta ecuación, encontramos que:
\[
2x = 10 \Rightarrow x = 5
\]

Por lo tanto, hay dos ciudades en la red donde llegan 5 carreteras a cada una.

\end{document}
