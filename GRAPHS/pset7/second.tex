\documentclass{article}
\usepackage[utf8]{inputenc}
\usepackage[spanish]{babel} % Soporte para caracteres en español
\usepackage{listings} % Para código
\usepackage{graphicx} % Para gráficos
\usepackage{tikz} % Para dibujar gráficas

\lstset{ % Configuración para el código
    language=Python,
    basicstyle=\ttfamily\small,
    frame=single,
    numbers=left,
    numberstyle=\tiny,
    xleftmargin=2em,
    flexclrs=true,
    breaklines=true,
    postbreak=\mbox{\textcolor{red}{$\hookrightarrow$}\space},
    extendedchars=true,
    literate={á}{{\'a}}1 {é}{{\'e}}1 {í}{{\'i}}1 {ó}{{\'o}}1 {ú}{{\'u}}1 {ñ}{{\~n}}1 {Á}{{\'A}}1 {É}{{\'E}}1 {Í}{{\'I}}1 {Ó}{{\'O}}1 {Ú}{{\'U}}1 {Ñ}{{\~N}}1
}

\title{Búsqueda en Gráficas y Árbol Generador}
\date{18 de Octubre de 2023}

\begin{document}
\maketitle

\section{Algoritmo de Búsqueda en Gráficas}

El algoritmo presentado a continuación es una implementación del visto en clase con Python para la búsqueda en gráficas. Se utiliza una lista \( C \) para almacenar los vértices que se van coloreando, y un conjunto \( colored \) para almacenar los vértices que ya han sido coloreados. El algoritmo recibe como parámetros la gráfica \( G \) y el vértice de inicio \( r \).

\begin{lstlisting}
def search_in_graph(G, start_vertex):
    C = []
    colored = set()
    i = 0

    # Selecciona el vértice de inicio y colorea de negro
    r = start_vertex
    colored.add(r)
    C.append(r)
    print(f"Inicializando con vértice {r}")

    while C:
        r = C[0]
        neighbors = G[r]
        found_uncolored = False
        
        for y in neighbors:
            if y not in colored:
                i += 1
                colored.add(y)
                C.append(y)
                print(f"Vértice {y} coloreado y agregado a C")
                found_uncolored = True
                break
        
        if not found_uncolored:
            C.remove(r)
            print(f"Vértice {r} removido de C")

    print(f"Terminado. Todos los vértices han sido coloreados.")
\end{lstlisting}

\section{Gráfica G}

La gráfica \( G \) se describe con los siguientes vértices y aristas:

\begin{tikzpicture}
    \node[draw, circle] (u) at (0,5) {u};
    \node[draw, circle] (v) at (4.5,5) {v};
    \node[draw, circle] (w) at (6,7) {w};
    \node[draw, circle] (x) at (3,7) {x};
    \node[draw, circle] (y) at (3,3) {y};
    \node[draw, circle] (z) at (6,3) {z};
    \node[draw, circle] (s) at (9,5) {s};
    
    \draw (u) -- (x);
    \draw (u) -- (y);
    \draw (v) -- (x);
    \draw (v) -- (w);
    \draw (v) -- (y);
    \draw (v) -- (z);
    \draw (w) -- (x);
    \draw (w) -- (s);
    \draw (w) -- (z);
    \draw (x) -- (v);
    \draw (y) -- (z);
    \draw (z) -- (s);
\end{tikzpicture}

\section{Árbol Generador con el Algoritmo de Búsqueda en Gráficas}

Para esto definimos otro método que recibe como parámetros la gráfica \( G \) y el vértice de inicio \( r \), y regresa el árbol generador \( T \) de la gráfica \( G \) con raíz en \( r \) que utiliza el algoritmo de búsqueda en gráficas.

\begin{lstlisting}
def print_generating_tree(G, start_vertex):
    # Utiliza el algoritmo BFS para obtener el árbol generador
    _, tree_edges = search_in_graph(G, start_vertex)
    
    # Imprime el árbol generador
    output = "Árbol Generador:\n"
    for edge in tree_edges:
        output += f"{edge[0]} -- {edge[1]}\n"
    return output.strip()

# Imprimir el árbol generador para la gráfica G con vértice de inicio u
print(print_generating_tree(G, "u"))
\end{lstlisting}

\section{Pasos del Algoritmo}

\begin{enumerate}
    \item Inicializando con el vértice u.
    \item Vértice x coloreado y agregado a C desde u.
    \item Vértice y coloreado y agregado a C desde u.
    \item Vértice u removido de C.
    \item Vértice v coloreado y agregado a C desde x.
    \item Vértice w coloreado y agregado a C desde x.
    \item Vértice x removido de C.
    \item Vértice z coloreado y agregado a C desde y.
    \item Vértice y removido de C.
    \item Vértice v removido de C.
    \item Vértice s coloreado y agregado a C desde w.
    \item Vértice w removido de C.
    \item Vértice z removido de C.
    \item Vértice s removido de C.
    \item Finalizado. Todos los vértices han sido coloreados.
\end{enumerate}


Una vez ejecutado el código anterior, se obtiene la siguiente salida:

\begin{lstlisting}
Árbol Generador:
u -- x
u -- y
x -- v
x -- w
y -- z
w -- s
\end{lstlisting}




\section{Gráfica del Árbol Generador}

\begin{tikzpicture}
    \node[draw, circle] (u) at (0,5) {u};
    \node[draw, circle] (x) at (2,7) {x};
    \node[draw, circle] (y) at (2,3) {y};
    \node[draw, circle] (v) at (4,7) {v};
    \node[draw, circle] (w) at (4,5) {w};
    \node[draw, circle] (z) at (4,3) {z};
    \node[draw, circle] (s) at (6,5) {s};
    
    \draw (u) -- (x);
    \draw (u) -- (y);
    \draw (x) -- (v);
    \draw (x) -- (w);
    \draw (y) -- (z);
    \draw (w) -- (s);
\end{tikzpicture}

Este árbol generador es el mismo que se obtiene al ejecutar el algoritmo de búsqueda en gráficas con raíz en \( u \).

\end{document}