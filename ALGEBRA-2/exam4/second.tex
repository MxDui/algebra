\documentclass[12pt]{article}

\usepackage{amsmath,amsfonts,amssymb}
\usepackage[utf8]{inputenc}
\usepackage[spanish]{babel}

\title{Examen 4 - Ejericio 2}
\author{David Rivera Morales}
\date{31 de Octubre de 2023}

\begin{document}

\maketitle

\section*{Problema}
Sea \( z = \sqrt[3]{2}(\cos(\pi/6) + i\sin(\pi/6)) \). Calcula \( z^3 \).

\section*{Solución}
Usando la Fórmula de De Moivre, sabemos que para cualquier número complejo en forma polar:

\[ z = r (\cos(\theta) + i \sin(\theta)) \]

Entonces:

\[ z^n = r^n (\cos(n\theta) + i \sin(n\theta)) \]

Dado nuestro número complejo \( z = \sqrt[3]{2}(\cos(\pi/6) + i\sin(\pi/6)) \), y aplicando la Fórmula de De Moivre para \( n = 3 \), obtenemos:

\[ z^3 = (\sqrt[3]{2})^3 (\cos(3\cdot\pi/6) + i \sin(3\cdot\pi/6)) \]
\[ z^3 = 2 (\cos(\pi/2) + i \sin(\pi/2)) \]
\[ z^3 = 2i \]

Por lo tanto, \( z^3 = 2i \).

\end{document}
