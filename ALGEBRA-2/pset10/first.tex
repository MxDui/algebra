\documentclass{article}
\usepackage[utf8]{inputenc}
\usepackage{amsmath, amssymb}

\title{Ejercicio 1}
\author{David Rivera Morales}
\date{19 de Noviembre 2023}

\begin{document}

\maketitle

\section{Problema}
Prueba que el polinomio
\[ 50x^7 + 49x^6 + 77x^5 + 35x^4 + 35x^2 + 21x + 14 \]
es irreducible en \( \mathbb{Z}[x] \).

\section{Definición de Irreducibilidad}
Un polinomio \( f(x) \) en \( A[x] \) se considera reducible si existen polinomios \( g(x) \) y \( h(x) \) en \( A[x] \), con grado menor al de \( f(x) \), tales que \( f(x) = g(x)h(x) \). En contraste, un polinomio es irreducible si no es reducible.

\section{Aplicación del Criterio de Eisenstein}
Para demostrar la irreducibilidad del polinomio \( P(x) \), utilizamos el Criterio de Eisenstein. Según este criterio, un polinomio es irreducible en \( \mathbb{Z}[x] \) si existe un número primo \( p \) que:

\begin{enumerate}
    \item Divide a todos los coeficientes del polinomio excepto al coeficiente líder.
    \item Su cuadrado \( p^2 \) no divide al término constante.
\end{enumerate}

\section{Aplicación del Criterio}
Aplicamos el Criterio de Eisenstein al polinomio \( P(x) \), eligiendo \( p = 7 \) como nuestro número primo. Observamos que:

\begin{itemize}
    \item \( 7 \) divide a 49, 77, 35, 21, y 14.
    \item \( 7 \) no divide a 50, el coeficiente líder de \( P(x) \).
    \item \( 49 \) (es decir, \( 7^2 \)) no divide a 14, el término constante.
\end{itemize}

\section{Conclusión}
Dado que el polinomio \( P(x) \) satisface ambas condiciones del Criterio de Eisenstein con \( p = 7 \), concluimos que es irreducible en \( \mathbb{Z}[x] \), de acuerdo con la definición de irreducibilidad proporcionada.

\end{document}
