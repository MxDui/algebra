\documentclass{article}
\usepackage{amsmath, amssymb}

\begin{document}
\title{Ejercicio 2}
\author{David Rivera Morales}
\date{31 de Agosto de 2023}
\maketitle

\section*{Proposición}
Si \(a, b, n \in \mathbb{N}\) y \(a \leq b\), entonces se cumple que:
\[a^n \leq b^n\]
para todo \(n \in \mathbb{N}\).

\section*{Demostración}

Para demostrar por inducción, seguimos un proceso de dos pasos:

\begin{enumerate}
    \item \textbf{Base de inducción:} Mostrar que la proposición es verdadera para el menor valor de \(n\) (en este caso, \(n = 1\)).
    \item \textbf{Paso inductivo:} Asumir que la proposición es verdadera para \(n = k\) y demostrar que también es verdadera para \(n = k + 1\).
\end{enumerate}

\noindent \textbf{Base de inducción:} \(n = 1\)

Si \(a \leq b\), entonces \(a^1 \leq b^1\), lo que implica \(a \leq b\). Esto es trivialmente cierto porque se nos da que \(a \leq b\).

\noindent \textbf{Paso inductivo:} Asumamos que la proposición es verdadera para \(n = k\). Es decir, asumamos que:
\[a^k \leq b^k\] (Hipótesis inductiva)

Necesitamos demostrar que:
\[a^{k+1} \leq b^{k+1}\]

Dado que \(a \leq b\), multiplicamos ambos lados de la hipótesis inductiva por \(a\):
\[a \cdot a^k \leq a \cdot b^k\]

Dado que \(a \leq b\), multiplicamos ambos lados de la hipótesis inductiva por \(b\):
\[a^k \cdot b \leq b^{k+1}\]

Combinando las dos inecuaciones:
\[a^{k+1} \leq a \cdot b^k \leq b^{k+1}\]

Por lo tanto, \(a^{k+1} \leq b^{k+1}\).

Hemos demostrado que si la proposición es verdadera para \(n = k\), entonces también es verdadera para \(n = k + 1\).

Por el principio de inducción matemática, la proposición \(a^n \leq b^n\) es verdadera para todo \(n \in \mathbb{N}\).

\end{document}
