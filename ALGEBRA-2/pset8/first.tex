\documentclass{article}
\usepackage{amsmath, amssymb}

\title{Ejercicio 1 - Tarea 8}
\author{David Rivera Morales}
\date{29 de Octubre de 2023}


\begin{document}

\maketitle

Dada la expresión \( z = \cos(\theta) + i \sin(\theta) \), se tiene que \( z = e^{i \theta} \).

\section*{a) Expresión para \( \frac{1}{{1 + z}} \)}

Para encontrar la expresión de \( \frac{1}{{1 + z}} \), se puede multiplicar y dividir por el conjugado del denominador:

\[
\frac{1}{{1 + e^{i \theta}}} \times \frac{{1 - e^{i \theta}}}{{1 - e^{i \theta}}} = 0 - \frac{\sin\left(\frac{\theta}{2}\right)}{\cos\left(\frac{\theta}{2}\right)}i
\]

\section*{b) Expresión para \( \frac{1}{{1 - z}} \)}

Para encontrar la expresión de \( \frac{1}{{1 - z}} \), se puede multiplicar y dividir por el conjugado del denominador:

\[
\frac{1}{{1 - e^{i \theta}}} \times \frac{{1 + e^{i \theta}}}{{1 + e^{i \theta}}} = 0 + \frac{\sin\left(\frac{\theta}{2}\right)}{\cos\left(\frac{\theta}{2}\right)}i
\]

\end{document}