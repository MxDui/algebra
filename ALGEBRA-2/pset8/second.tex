\documentclass{article}
\usepackage{amsmath, amssymb}

\title{Ejercicio 2 - Tarea 8}
\author{David Rivera Morales}
\date{29 de Octubre de 2023}

\begin{document}

\maketitle

\section*{Problema}

Usando la expresión de la Fórmula de De Moivre, encuentre las raíces cuartas de \(-16\).

\section*{Solución}

La fórmula de De Moivre nos dice que si \( z = r(\cos \theta + i \sin \theta) \), entonces:

\[
z^n = r^n (\cos(n\theta) + i \sin(n\theta))
\]

Para representar \(-16\) en su forma polar, observamos que su magnitud es 16 y su ángulo es \(\pi\). Por lo tanto:

\[
-16 = 16(\cos \pi + i \sin \pi)
\]

Buscamos \( z \) tal que:

\[
z^4 = 16(\cos \pi + i \sin \pi)
\]

Usando la fórmula de De Moivre y teniendo en cuenta que \(z^n = r^n(\cos(n\theta) + i \sin(n\theta))\):

\[
z = 16^{\frac{1}{4}} \left( \cos\left(\frac{\pi + 2\pi k}{4}\right) + i \sin\left(\frac{\pi + 2\pi k}{4}\right) \right)
\]

El factor \(2\pi k\) se introduce para considerar todas las raíces \(n\)-ésimas. Para raíces cuartas, \( k = 0, 1, 2, 3 \).

Vamos a calcular cada raíz:

\begin{enumerate}
    \item Para \( k = 0 \):
    \[
    z = 16^{\frac{1}{4}} \left( \cos\left(\frac{\pi}{4}\right) + i \sin\left(\frac{\pi}{4}\right) \right) = \sqrt{2} + \sqrt{2}i
    \]
    
    \item Para \( k = 1 \):
    \[
    z = 16^{\frac{1}{4}} \left( \cos\left(\frac{3\pi}{4}\right) + i \sin\left(\frac{3\pi}{4}\right) \right) = -\sqrt{2} + \sqrt{2}i
    \]
    
    \item Para \( k = 2 \):
    \[
    z = 16^{\frac{1}{4}} \left( \cos\left(\frac{5\pi}{4}\right) + i \sin\left(\frac{5\pi}{4}\right) \right) = -\sqrt{2} - \sqrt{2}i
    \]
    
    \item Para \( k = 3 \):
    \[
    z = 16^{\frac{1}{4}} \left( \cos\left(\frac{7\pi}{4}\right) + i \sin\left(\frac{7\pi}{4}\right) \right) = \sqrt{2} - \sqrt{2}i
    \]
\end{enumerate}

Por lo tanto, las raíces cuartas de \(-16\) son:

\begin{enumerate}
    \item \( \sqrt{2} + \sqrt{2}i \)
    \item \( -\sqrt{2} + \sqrt{2}i \)
    \item \( -\sqrt{2} - \sqrt{2}i \)
    \item \( \sqrt{2} - \sqrt{2}i \)
\end{enumerate}

\end{document}
