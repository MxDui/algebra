\documentclass{article}
\usepackage{amsmath, amssymb}

\begin{document}

\title{Álgebra Superior II - Tarea 6 - Ejericio 1}
\author{David Rivera Morales}
\date{\today} 

\maketitle

\section*{Teorema}
Sean \( a, b \in \mathbb{Z} \) y \( p \) un número primo. Si \( a^2 \equiv b^2 \) (mod \( p \)), entonces \( a \equiv \pm b \) (mod \( p \)).

\section*{Demostración}

Dado que \( a^2 \equiv b^2 \) (mod \( p \)), podemos reescribirlo como:
\[ a^2 - b^2 \equiv 0 \] (mod \( p \))

Factorizando la expresión del lado izquierdo:
\[ (a + b)(a - b) \equiv 0 \] (mod \( p \))

Dado que \( p \) es primo, la única manera de que un producto sea congruente a 0 módulo \( p \) es que al menos uno de los factores sea congruente a 0 módulo \( p \). Esto implica que:
\[ a + b \equiv 0 \] (mod \( p \))
o
\[ a - b \equiv 0 \] (mod \( p \))

De la primera ecuación, obtenemos:
\[ a \equiv -b \] (mod \( p \))

De la segunda ecuación, obtenemos:
\[ a \equiv b \] (mod \( p \))

Por lo tanto, si \( a^2 \equiv b^2 \) (mod \( p \)), entonces \( a \equiv \pm b \) (mod \( p \)).

\end{document}
