\documentclass{article}
\usepackage{amsmath, amssymb, amsthm}

\begin{document}

\section*{Demostración}
Dado que \( bz \equiv 0 \) (mod \( a \)), podemos afirmar que \( a \) divide \( bz \).

Queremos probar que \( z \equiv 0 \) (mod \( a \)) utilizando el Lema de Euclides.

\textbf{Lema de Euclides:} Si \( m \) y \( n \) son enteros coprimos y \( m \) divide \( pk \) para algún entero \( k \), entonces \( m \) divide \( k \).

\textit{Estrategia de demostración:} La elección de utilizar el Lema de Euclides se basa en su utilidad al manejar situaciones en las cuales tenemos dos números que son coprimos y uno de ellos divide un producto. Dado que \( (a, b) = 1 \) y \( bz \equiv 0 \) (mod \( a \)), se ajusta a la situación en la que el Lema de Euclides puede ser aplicado.

\begin{proof}
Dado que \( bz \equiv 0 \) (mod \( a \)), sabemos que \( a \) divide \( bz \).

Por el Lema de Euclides, dado que \( a \) y \( b \) son coprimos y \( a \) divide \( bz \), entonces \( a \) también divide \( z \).

Si \( a \) divide \( z \), entonces \( z \equiv 0 \) (mod \( a \)). 
\end{proof}

\end{document}
