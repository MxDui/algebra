\documentclass{article}
\usepackage{amsmath, amssymb, amsthm}

\author{David Rivera Morales}
\title{Demostración}
\date{1 de octubre de 2023}


\begin{document}

\maketitle
Sean \( a, b, z \in \mathbb{Z} \) números enteros tales que \( (a, b) = 1 \) y \( bz \equiv 0 \) (mod \( a \)). Dado que \( bz \equiv 0 \) (mod \( a \)), podemos afirmar que \( a \) divide \( bz \).

\textbf{P.D} Demostrar que \( z \equiv 0 \) (mod \( a \)).

\textbf{Lema de Euclides:} Si \( m \) y \( n \) son enteros coprimos y \( m \) divide \( pk \) para algún entero \( k \), entonces \( m \) divide \( k \).

\begin{Demostración}
\begin{enumerate}
    \item Partimos de la suposición de que \( bz \equiv 0 \) (mod \( a \)). Esto implica que \( a \) divide \( bz \).
    \item Dado que \( (a, b) = 1 \) (lo que significa que \( a \) y \( b \) son coprimos), y usando el Lema de Euclides, si \( a \) divide el producto \( bz \), entonces \( a \) debe dividir \( z \).
    \item Usando la definición de números coprimos, sabemos que \( \text{mcd}(a, b) = 1 \). Dado que \( bz \equiv 0 \) (mod \( a \)), esto significa que \( a \) divide \( bz \). Por el Lema de Euclides, dado que \( \text{mcd}(a, b) = 1 \) y \( a \) divide \( bz \), \( a \) también divide \( z \).
    \item Finalmente, si \( a \) divide \( z \), entonces por definición tenemos que \( z \equiv 0 \) (mod \( a \)).
\end{enumerate}
Por lo tanto, hemos demostrado que si \( bz \equiv 0 \) (mod \( a \)), entonces \( z \equiv 0 \) (mod \( a \)).
\end{Demostración}

\end{document}
