\documentclass{article}
\usepackage{amsmath, amssymb}

\begin{document}

\title{Divisibilidad del Producto de Tres Números Consecutivos}
\maketitle

\section*{Planteamiento del Problema}
Demuestra que el producto de tres números enteros consecutivos es divisible entre 6, es decir, \(\forall n \in \mathbb{N}\) 
\[ 6|n(n + 1)(n + 2) \]

\section*{Demostración}

Para demostrar esto, vamos a utilizar una combinación lineal (o ecuación diofántica). Dado que \( n \), \( n+1 \), y \( n+2 \) son tres números consecutivos, al menos uno de ellos es divisible por 2 y al menos uno de ellos es divisible por 3. Podemos expresar este hecho mediante una combinación lineal de \( n \), \( n+1 \), y \( n+2 \) que sea igual a 2 o 3.

Para demostrar que \( n(n+1)(n+2) \) es divisible por 6, debemos mostrar que hay coeficientes enteros \( a \), \( b \), y \( c \) tales que:
\[ an + b(n+1) + c(n+2) = 6 \]

Para el caso de divisibilidad por 2:
\begin{enumerate}
    \item Si \( n \) es par, entonces \( a = 1 \), \( b = 0 \), \( c = 0 \) satisface la ecuación.
    \item Si \( n+1 \) es par, entonces \( a = 0 \), \( b = 1 \), \( c = 0 \) satisface la ecuación.
    \item Si \( n+2 \) es par, entonces \( a = 0 \), \( b = 0 \), \( c = 1 \) satisface la ecuación.
\end{enumerate}

Para el caso de divisibilidad por 3:
\begin{enumerate}
    \item Si \( n \) es divisible por 3, entonces \( a = 1 \), \( b = 0 \), \( c = 0 \) satisface la ecuación.
    \item Si \( n+1 \) es divisible por 3, entonces \( a = 0 \), \( b = 1 \), \( c = 0 \) satisface la ecuación.
    \item Si \( n+2 \) es divisible por 3, entonces \( a = 0 \), \( b = 0 \), \( c = 1 \) satisface la ecuación.
\end{enumerate}

Por lo tanto, hemos demostrado que al menos uno de los números es divisible por 2 y al menos uno de los números es divisible por 3, lo que implica que el producto \( n(n+1)(n+2) \) es divisible por 6.

\end{document}
