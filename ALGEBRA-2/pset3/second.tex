\documentclass{article}
\usepackage{amsmath}

\begin{document}

\section*{Máximo Común Divisor usando el Algoritmo de Euclides y el Algoritmo de División}

\subsection*{Para \(a = 434\) y \(b = 31\)}
\begin{align*}
434 & = 31 \times 14 + 0 &\quad \text{(Dividimos 434 entre 31, el residuo es 0. Según el algoritmo, hacemos } a = b \text{ y } b = \text{residuo)} \\
\therefore \text{MCD}(434,31) & = 31 &\quad \text{(Dado que el residuo es 0, } b = 31 \text{ es el MCD)}
\end{align*}

\subsection*{Para \(a = 8611\) y \(b = -17\)}
Dado que el MCD no cambia si uno de los números es negativo, tomamos \(b = 17\).
\begin{align*}
8611 & = 17 \times 506 + 5 &\quad \text{(Dividimos 8611 entre 17, el residuo es 5. Según el algoritmo, hacemos } a = b \text{ y } b = \text{residuo)} \\
17 & = 5 \times 3 + 2 &\quad \text{(Dividimos 17 entre 5, el residuo es 2. Según el algoritmo, hacemos } a = b \text{ y } b = \text{residuo)} \\
5 & = 2 \times 2 + 1 &\quad \text{(Dividimos 5 entre 2, el residuo es 1. Según el algoritmo, hacemos } a = b \text{ y } b = \text{residuo)} \\
2 & = 1 \times 2 + 0 &\quad \text{(Dividimos 2 entre 1, el residuo es 0. Según el algoritmo, hacemos } a = b \text{ y } b = \text{residuo)} \\
\therefore \text{MCD}(8611,-17) & = 1 &\quad \text{(Dado que el residuo es 0, } b = 1 \text{ es el MCD)}
\end{align*}

\end{document}
