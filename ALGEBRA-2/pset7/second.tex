\documentclass{article}
\usepackage{amsmath, amssymb}

\title{Ejercicio 2 - Tarea 7}
\date{22 de octubre de 2023}

\begin{document}

\maketitle
Dado:
\[ z_1 = 1 - i \]
\[ z_2 = -2 + 4i \]
\[ z_3 = \sqrt{3} - 2i \]

Encuentra el valor de las siguientes expresiones:

1) \[ \left| \frac{z_1 + z_2 + i}{z_1 - z_2 - 1} \right| \]

Solución:

Primero calculamos el numerador:
\[ (1 - i) + (-2 + 4i) + i = -1 + 4i \]

Luego, el denominador:
\[ (1 - i) - (-2 + 4i) - 1 = 2 - 5i \]

Finalmente, la expresión se reduce a:
\[ \left| \frac{-1 + 4i}{2 - 5i} \right| \]

Para simplificar el cálculo, multiplicamos y dividimos por el conjugado del denominador:
\[ = \left| \frac{-1 + 4i}{2 - 5i} \times \frac{2 + 5i}{2 + 5i} \right| \]

Esto nos da:
\[ \left| -0.7586 + 0.1034i \right| \]
\[ \approx 0.766 \]

2) \( \text{Im}\left( \frac{z_1 z_2}{z_3} \right) \)

Solución:

Primero, multiplicamos \( z_1 \) y \( z_2 \):
\[ (1 - i)(-2 + 4i) = -2 + 2i + 4i + 4 = 2 + 6i \]

Luego, dividimos este resultado por \( z_3 \):
\[ \frac{2 + 6i}{\sqrt{3} - 2i} \]
\[ = -1.2194 + 2.0560i \]

De este resultado, tomamos la parte imaginaria:
\[ \text{Im}\left( -1.2194 + 2.0560i \right) \]
\[ \approx 2.056 \]

Por lo tanto, los valores de las expresiones son:

1) \( \approx 0.766 \)
2) \( \approx 2.056 \)

\end{document}
