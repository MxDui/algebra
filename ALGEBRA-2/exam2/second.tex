\documentclass{article}
\usepackage[spanish]{babel}
\usepackage{amsmath}

\begin{document}

\title{Problema de la Póliza de Seguro y su Resolución}
\author{David Rivera Morales}
\date{2023-09-21}

\maketitle

El enunciado indica que:
\begin{itemize}
\item El valor de la póliza es de $p$ pesos con $c$ centavos.
\item Por error, al asegurado se le da $c$ pesos y $p$ centavos.
\item El asegurado gasta 23 centavos.
\item En ese momento tiene $2p$ pesos y $2c$ centavos.
\end{itemize}

Podemos traducir esta información a ecuaciones, llamando $V$ al valor original de la póliza en centavos. Entonces:
\begin{equation}
V = 100c + p
\end{equation}

Al asegurado se le dio $c$ pesos (equivalentes a $100c$ centavos) y $p$ centavos. Tras gastar 23 centavos, le quedaron:
\begin{equation}
(100c + p) - 23 \text{ centavos}
\end{equation}

El enunciado indica que en ese momento tenía $2p$ pesos y $2c$ centavos, esto es:
\begin{equation}
200p + 2c \text{ centavos}
\end{equation}

Igualando las expresiones obtenemos:
\begin{equation}
100c + p - 23 = 200p + 2c
\end{equation}

Despejando $c$:
\begin{align}
100c - 23 &= 199p + 2c \\
98c &= 199p + 23 \\
c &= \frac{199p + 23}{98}
\end{align}

Para obtener una solución entera, buscamos un valor de $p$ tal que $199p + 23$ sea divisible entre 98 (es decir, entre 2 y 49).

Probando con $p = 47$:
\begin{align*}
199(47) + 23 &= 9453 \\
&= 94 \times 100 + 53
\end{align*}

Sustituyendo en la ecuación para $c$:
\begin{equation}
c = \frac{199(47) + 23}{98} = 96
\end{equation}

Por lo tanto, la suma asegurada es 47 pesos con 96 centavos

\end{document}
