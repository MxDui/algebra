\documentclass{article}
\usepackage{amssymb}

\begin{document}

\title{My First LaTeX Document}
\author{David Rivera Morales}
\date{\today}
\maketitle

\section{Prove that for $a,b,c,d \in \mathbb{N}$, if $a < c$, then $\exists t \in \mathbb{N}$ such that $a + t = c$.}

\begin{proof}
Assume $a < c$. Then $\exists t \in \mathbb{N}$ such that $a + t = c$. Also $b < d$ and $\exists s \in \mathbb{N}$ such that $b + s = d$.
\item $a + t + (b + s) = c + d$.
\item $(a + b) + (t + s) = c + d$.
\item $c + d = (a + b) + (t + s)$.
\item By definition of order 
In the context of natural numbers, the term "order" typically refers to the relation of "less than or equal to". We say that a natural number $a$ is less than or equal to another natural number $b$ if $a$ can be obtained from $b$ by subtracting a non-negative integer. This relation is denoted by $a \leq b$. 
Formally, we can define the order relation on natural numbers as follows:
For any natural numbers $a$ and $b$, we say that $a$ is less than or equal to $b$, denoted by $a \leq b$, if there exists a natural number $c$ such that $a + c = b$.$=> a + b < c + d$.
\item $a + b < c + d$.

\item $a < c$ and $b < d$ $=>$ $a + b < c + d$.

\end{proof}

% necessary math concepts
\begin{itemize}
    \item {\bf Natural Numbers ($\mathbb{N}$)}: The set of natural numbers is usually denoted by $\mathbb{N}$. It includes all non-negative integers (0, 1, 2, 3, ...). This set is fundamental in mathematics, forming the basis of number theory and providing the most basic numeral system for counting and ordering objects.
    
    \item {\bf Inequalities}: Inequalities are relations between two values that may not be equal. The notation $a < b$ signifies that $a$ is less than $b$, while $a > b$ means that $a$ is greater than $b$. Inequalities express a fundamental concept of order among numbers, often in the context of real numbers or integers.
    
    \item {\bf Existential Quantifier ($\exists$)}: The existential quantifier is a symbol used in logic and mathematics to denote that there is at least one thing (or number, in our case) that satisfies a given condition. For example, the statement "$\exists x \in \mathbb{N}$ such that $x > 2$" means "There is a natural number that is greater than 2".
    
    \item {\bf Addition}: Addition is a basic arithmetic operation representing the total amount of objects together in a collection. It is a binary operation, meaning it takes two inputs (the addends) and produces one output (the sum). The operation is commutative and associative in the set of natural numbers and has a neutral element, 0, such that $a + 0 = a$ for any $a \in \mathbb{N}$.
    
    \item {\bf Associative Property of Addition}: This property states that the way in which numbers are grouped in an addition operation does not change the result. That is, for any numbers $a$, $b$, and $c$, we have $(a + b) + c = a + (b + c)$. This property is fundamental in arithmetic and algebra and holds in many mathematical structures, including natural numbers, integers, real numbers, and complex numbers.
    
    \item {\bf Order Relation in Natural Numbers}: The order relation among natural numbers is usually expressed with the symbols $<$ and $>$ (less than and greater than, respectively) or $\leq$ and $\geq$ (less than or equal to and greater than or equal to, respectively). The order relation has important properties such as transitivity ($a < b$ and $b < c$ imply $a < c$), reflexivity ($a \leq a$ for all $a$), and antisymmetry (if $a \leq b$ and $b \leq a$, then $a = b$).
    
    \item {\bf Implication ($\Rightarrow$)}: In logic and mathematics, implication is a fundamental logical operation. It relates two statements in such a way that if the first statement is true, then the second statement is also guaranteed to be true. It is denoted by $\Rightarrow$. For example, the statement "$a < b \Rightarrow a + c < b + c$" means that if $a$ is less than $b$, then $a$ plus any number $c$ is less than $b$ plus $c$.
\end{itemize}

\section{Prove that for all $n \in \mathbb{N}$, 0^2 + 1^2 + 2^2 + ... + n^2 = \frac{n(n+1)(2n+1)}{6}$}

\begin{proof}

    \item Let A={$n \in \mathbb{N}$ | $0^2 + 1^2 + 2^2 + ... + n^2 = \frac{n(n+1)(2n+1)}{6}$}.
    \item Base Case: $n = 0 with 0 \in A$.
    \item Proof $0 \in A$.
    \item $0^2 = \frac{0(0+1)(2(0)+1)}{6}$.
    \item $0 = \frac{0(1)(1)}{6}$.
    \item $0 = \frac{0}{6}$.
    \item $0 = 0$.
    \item Induction Hypothesis: $0^2 + 1^2 + 2^2 + ... + n^2 = \frac{n(n+1)(2n+1)}{6}$.
    \item Inductive Step: $n+1$.
    \item $0^2 + 1^2 + 2^2 + ... + n^2 + (n+1)^2 = \frac{(n+1)((n+2))(2(n+1)+1)}{6}$.
    \item $0^2 + 1^2 + 2^2 + ... + n^2 + (n+1)^2 = \frac{(n+1)(n+2)(2n+3)}{6}$.

    \item $0^2 + 1^2 + 2^2 + ... + n^2 + (n+1)^2 = \frac{(n)(n+1)(2n+1){6}} + (n+1)^2$.
    \item $0^2 + 1^2 + 2^2 + ... + n^2 + (n+1)^2 = \frac{(n)(n+1)(2n+1+6(n+1)^2)}{6}$.
    \item $0^2 + 1^2 + 2^2 + ... + n^2 + (n+1)^2 = \frac{(n+1)(n(2n+1)+6(n+1))}{6}$.
    \item $0^2 + 1^2 + 2^2 + ... + n^2 + (n+1)^2 = \frac{(n+1)(2n^2+n+6n+6)}{6}$.
    \item $0^2 + 1^2 + 2^2 + ... + n^2 + (n+1)^2 = \frac{(n+1)(2n^2+7n+6)}{6}$.
    \item $0^2 + 1^2 + 2^2 + ... + n^2 + (n+1)^2 = \frac{(n+1)(n+2)(2n+3)}{6}$.
    \item $0^2 + 1^2 + 2^2 + ... + n^2 + (n+1)^2 = \frac{(n+1)((n+2))(2(n+1)+1)}{6}$.


\end{proof}

\begin{itemize}

    \item {\bf Mathematical Induction}: This is a method of mathematical proof typically used to establish a given statement for all natural numbers. It involves two steps: the base case, which proves the statement for the initial value (often 0 or 1), and the inductive step, where one assumes the statement for some arbitrary natural number $n$, and then proves it for $n+1$. If these two steps are successfully completed, then the statement is true for all natural numbers.
    
    \item {\bf Summation of squares ($0^2 + 1^2 + 2^2 + ... + n^2$)}: This is a specific type of series which involves the sum of the squares of the first $n$ natural numbers. These series appear in a variety of mathematical areas, including number theory and combinatorics. 
    
    \item {\bf Algebraic Manipulation and Factoring}: In the inductive step of the proof, we perform algebraic manipulations on the formula, including factoring. Factoring is the process of expressing a polynomial as the product of its factors. Factoring is often used to simplify mathematical expressions and solve equations. For instance, we have factored $(n+1)(n+2)(2n+3)$ from $(n+1)(2n^2+7n+6)$ to get to the result. 
    
    \item {\bf The Formula $\frac{n(n+1)(2n+1)}{6}$}: This is the closed form for the sum of the squares of the first $n$ natural numbers. A "closed form" is an expression that can be evaluated in a finite number of operations. The derivation of such a formula typically involves techniques from calculus, combinatorics, or other areas of mathematics. Here, the proof that this formula holds for all natural numbers $n$ is given by induction.
    
\end{itemize}

\section{Prove that for all $\sum_{i=1}^{n} 2(3^{i-1}) = 3^n - 1$}

\begin{proof}
    \item Let A={$n \in \mathbb{N}$ | $\sum_{i=1}^{n} 2(3^{i-1}) = 3^n - 1$}.
    \item Base Case: $n = 1 with 1 \in A$.
    \item Proof $1 \in A$.
    \item $\sum_{i=1}^{1} 2(3^{i-1}) = 3^1 - 1$.
    \item $2(3^{1-1}) = 3^1 - 1$.
    \item $2(3^0) = 3^1 - 1$.
    \item $2(1) = 3^1 - 1$.
    \item $2 = 3 - 1$.
    \item $2 = 2$.
    \item Induction Hypothesis: $\sum_{i=1}^{n} 2(3^{i-1}) = 3^n - 1$.
    \item Inductive Step: $n+1$.
    \item $\sum_{i=1}^{n+1} 2(3^{i-1}) = 3^{n+1} - 1$.
    \item $\sum_{i=1}^{n+1} 2(3^{i-1}) = \sum_{i=1}^{n} 2(3^{i-1}) + 2(3^{n+1-1})$ 
    \item $\sum_{i=1}^{n+1} 2(3^{i-1}) = 3^n - 1 + 2(3^{n})$.
    \item $\sum_{i=1}^{n+1} 2(3^{i-1}) = 3^n (2+1) - 1$.
    \item $\sum_{i=1}^{n+1} 2(3^{i-1}) = 3^n (3) - 1$.
    \item $\sum_{i=1}^{n+1} 2(3^{i-1}) = 3^{n+1} - 1$.

    \end{proof}
     

    

\end{document}