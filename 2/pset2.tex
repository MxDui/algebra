\documentclass{article}
\usepackage[spanish]{babel}
\usepackage[utf8]{inputenc}
\usepackage{amsmath, amssymb}

\begin{document}

\textbf{Problema:}

Sea \( (R, +, \cdot) \) un anillo. Muestra que \( R = \{0\} \) si y solo si el neutro multiplicativo es igual al neutro aditivo.

\textbf{Solución:}

Dado un anillo \( (R, +, \cdot) \), denotemos el neutro aditivo por \( 0 \) y el neutro multiplicativo (si existe) por \( 1 \).

1. (\( \Rightarrow \)): Supongamos que \( R = \{0\} \), el anillo trivial. Entonces, \( 0 \) actúa tanto como el neutro aditivo como el neutro multiplicativo, ya que no hay otros elementos en el anillo. Por lo tanto, en este caso, el neutro multiplicativo es igual al neutro aditivo.

2. (\( \Leftarrow \)): Supongamos que el neutro multiplicativo es igual al neutro aditivo, es decir, \( 1 = 0 \). Ahora, considera cualquier elemento \( a \) en \( R \). Tenemos:
\[
a = a \cdot 1 = a \cdot 0 = 0 
\]
Donde \( a \cdot 1 = a \) porque \( 1 \) es el neutro multiplicativo y \( a \cdot 0 = 0 \) porque en cualquier anillo, la multiplicación por el neutro aditivo da como resultado el neutro aditivo. Por lo tanto, cada elemento de \( R \) es \( 0 \), lo que implica que \( R = \{0\} \).

Por lo tanto, hemos demostrado que \( R = \{0\} \) si y solo si el neutro multiplicativo es igual al neutro aditivo.

\end{document}
