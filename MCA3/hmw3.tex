\documentclass{article}
\usepackage[utf8]{inputenc}
\usepackage{amsmath}
\usepackage{amsfonts}
\usepackage{amssymb}

\title{Tarea 3 - MCA III}
\author{David Rivera Morales}
\date{}

\begin{document}

\maketitle

\section*{Preguntas}

\subsection*{Pregunta 1}
\begin{itemize}
    \item \textbf{Hallar la longitud total del alambre representado por la curva \( C = C1 \cup C2 \), donde \( C1 \) es parte de la parábola \( y = x^2 \) y \( C2 \) es un segmento de recta.}
    \begin{itemize}
        \item \textbf{Caso C1:} La parametrización es \( t = x \to y = t^2 \), entonces los puntos \((1, 1, 0)\) serán \( (t, t^2, 0) \). Derivando tenemos que \(\frac{dx}{dt} = x' = 1 \), \(\frac{dy}{dt} = y' = 2t \) y \(\frac{dz}{dt} = z' = 0 \). Con las derivadas anteriores y sabiendo que va de 0 a 1 podemos plantear:
        \[
        \int_{0}^{1} \sqrt{x'^{2} + y'^{2} + z'^{2}} \, dt = \int_{0}^{1} \sqrt{1 + 4t^{2}} \, dt
        \]
        \[
        = \frac{2t (1 + 4t^{2})^{\frac{1}{2}}}{4} + \frac{\ln \left(2t + \sqrt{1 + 4t^{2}}\right)}{4} \Big|_{0}^{1} 
        \]
        \[
        = \frac{2(1 + 4)^{\frac{1}{2}}}{4} + \frac{\ln \left(2 + \sqrt{1 + 4}\right)}{4} = 1.478 = C1
        \]
        
        \item \textbf{Caso C2:} Planteamos a \( z \) como \( t \) y la parametrización será \( (1,1,t) \).
        \[
        \text{Derivadas} \quad \frac{dx}{dt} = 0, \quad \frac{dy}{dt} = 0, \quad \frac{dz}{dt} = 1
        \]
        \[
        \text{La integral que va de 0 a 1 es} \quad \int_{0}^{1} \sqrt{0 + 0 + 1} \, dt = 1
        \]

        \item \textbf{Sumando las longitudes de C1 y C2:} \quad \( 1.478 + 1 = 2.478 \)

        \item \textbf{Por lo tanto, la longitud total del alambre es de 2.478.}
    \end{itemize}

    \item \textbf{Si la densidad en cada punto del alambre está dada por la función \( \rho(x, y, z) = x + \sqrt{y} - z^2 \), hallar la masa total del alambre.}
     \begin{itemize}
        \item La masa total del alambre, calculada como la suma de las masas de \( C1 \) y \( C2 \), es \( 3.364 \). En la siguiente pregunta se encuentran los cálculos.
    \end{itemize}
    
    \item \textbf{Hallar la densidad promedio del alambre.}
    \begin{itemize}
        \item La fórmula del valor promedio es \( \frac{\int_{C} \rho(x, y, z) \, ds}{\int_{C} ds} \).
        \item Sabiendo que la longitud del arco calculada es \( \int_{C} ds = 2.478 \), se calcula \( \frac{\int_{C} \rho(x, y, z) \, ds}{2.478} \).
        \item Entonces para C1 y C2 parametrizamos
        \[
            C1 \to t=x , \frac{dx}{dt} = x' = 1, \frac{dy}{dt} = y' = 2t, \frac{dz}{dt} = 0
            \]
            \item La función ahora parametrizada será:
            \[
                P(x,y,z)= x + sqrt{y} -z^{2}
                \]
                \[
                    P(t,t^{2},0) = t + \sqrt{t^{2}} - 0 = t + t - 0 = 2t
                    \]
                    \[
                        \int_{0}^{1} 2t \sqrt{1 + 4t^{2}} \, dt , u = 1 + 4t^{2}, du = 8t \, dt
                        \]
                        \[
                            \int_{0}^{1} \frac{1}{4} u^{\frac{1}{2}} \, du = \frac{1}{4} \int_{0}^{1} u^{\frac{1}{2}} du = \frac{1}{6} u^{\frac{3}{2}} \Big|_{0}^{1}
                            \]
                            \[
                                 = \frac{1}{6} (1+4 t^{2})^{\frac{3}{2}} \Big|_{0}^{1} = \frac{(1+4t^{2})\sqrt{1+4t^{2}}}{6} \Big|_{0}^{1} = \frac{5\sqrt{5}}{6} - \frac{1}{6} =  1.697 = C1
                                 \]
        \[
            C2 \to \frac{dx}{dt} = 0, \frac{dy}{dt} = 0, \frac{dz}{dt} = 1 .
            \]
            \item  Sustituimos en la función P(x,y,z) y  en la parametrización de C2:
            \[
                P(x,y,z) = x + \sqrt{y} - z^{2} 
                \]
                \[
                    P(1,1,t) = 2 - t^{2}
                    \]
                    \item Planteamos la integral de 0 a 1:
                    \[
                        \int_{0}^{1} 2 - t^{2} \, dt = 2t - \frac{t^{3}}{3} \Big|_{0}^{1} = 2 - \frac{1}{3} = \frac{5}{3} = C2
                        \]
                        \item Sumando C1 y C2:
                        \[
                            1.697 + \frac{5}{3} = 3.364
                            \]
                            \item Por lo tanto, la masa total del alambre es de 3.364.
                            \item Para hallar la densidad promedio se divide la masa total entre la longitud total del alambre:
                            \[
                                \frac{3.364}{2.478} = 1.358
                                \]
                                \item Por lo tanto, la densidad promedio del alambre es de 1.357.


                      
                
    \end{itemize}
\end{itemize}

\subsection*{Pregunta 2}
\begin{itemize}
    \item \textbf{Si con 1 L de pintura se pueden cubrir 100 m\(^2\), ¿cuántos litros de pintura se necesitan para pintar ambos lados de la valla?}
    \begin{itemize}
        \item La base de la valla está descrita por \( x = 10 \cos t \), \( y = 10 \sin t \), según la parametrización \( r(t) = 10 \cos t \mathbf{i} + 10 \sin t \mathbf{j} \).
        \item Entonces \(\frac{dx}{dt} = -10 \sin t \), \(\frac{dy}{dt} = 10 \cos t \), y planteando una integral y sabiendo que \(z(x, y) = 0.01(x^2 - y^2) + 4\), se tiene:
        \item \to\[ \int_{0}^{2\pi} \left(0.1( \cos^2 t - \sin^2 t) + 4\right) \sqrt{(-10 \sin t)^2 + (10 \cos t)^2 + 1} \, dt \]
        \item \to\[ \int_{0}^{2\pi} \left(0.1( \cos^2 t -  \sin^2 t) + 4\right) \sqrt{100} \, dt \]
        \item \to\[ \int_{0}^{2\pi} \left(0.1( \cos^2 t -  \sin^2 t) + 4\right) 10 \, dt \]
        \item \to\[ \int_{0}^{2\pi} \left( \cos^2 t -  \sin^2 t + 40\right) \, dt \]
        \item \to\[ (\frac{\cos t \sin t}{2} - \frac{\sin t \cos t}{2} + 40t) \Big|_{0}^{2\pi} \]
        \item \to\[ 80\pi \]
        \item \to \[ \frac{80\pi m^2}{100 m^2/L} \]
        \item \to \[ \frac{4}{5} \pi L \] Para un lado de la valla.
        \item \to \[ \frac{8}{5} \pi L \] Para ambos lados de la valla.

        
        
    \end{itemize}
    
    \item \textbf{¿Cuál es la altura promedio que tiene la valla?}
    \begin{itemize}
        \item Utilizando el valor de \( \int_{C} P(x, y, z) \, dz \) y los derivados \( x' = -10 \sin t \), \( y' = 10 \cos t \), \( z' = 0 \), se calcula la altura promedio con la fórmula:
        \[ \frac{\int_{C} P(x, y, z) \, dz}{\int_{C} ds} \] \to \[ \frac{\int_{0}^{2\pi} (0.01 (10 \cos^2 t^ - 10 \sin^2 t)+4) \sqrt{(-10 \sin t)^2 + (10 \cos t)^2 + 1} \, dt}{\int_{0}^{2\pi} \sqrt{(-10 \sin t)^2 + (10 \cos t)^2 } \, dt} \] \to         \[ \frac{80\pi m^2}{\int_{0}^{2\pi} \sqrt{100(\cos^2 t + \sin^2 t)} \, dt} \]   \end{itemize}
        \[ \frac{80\pi m^2}{\int_{0}^{2\pi} 10 \, dt m} \] \to \[ \frac{80\pi m^2}{20\pi m} \] \to \[ 4 m \]
        \item Por lo tanto, la altura promedio de la valla es de 4 metros.
    
    \end{itemize}

    \item \textbf{¿Cuál es la longitud de la valla entre los puntos (5\sqrt{3},5) y (5\sqrt{2},5\sqrt{2})?}
    \begin{itemize}
        \item Recordemos que \(z(x, y) = 0.01(x^2 - y^2) + 4\), entonces parametrizando 
        \[ r(t) = (1-t)(5\sqrt{3},5) + t(5\sqrt{2},5\sqrt{2}) \]
        \[ = (5\sqrt{3} - 5\sqrt{3}t, 5 - 5t) + (5\sqrt{2}t, 5\sqrt{2}t) \]
        \[ = 5 (\sqrt{3} - \sqrt{3}t + \sqrt{2}t), 5(1 + \sqrt{2}t - t) \]
        \item Planteamos la siguiente integral:
        \[ \int_{0}^{1} \frac{(5\sqrt{3} - 5\sqrt{3}t + 5\sqrt{2}t + 5 + 5\sqrt{2}t - 5t)}{\sqrt{(5\sqrt{2} - 5\sqrt{3})^2 + (5\sqrt{2} - 5)^2 }} \, dt \]
        \[ = \int_{0}^{1} \sqrt{(5\sqrt{2} - 5\sqrt{3})^2 + (5\sqrt{2} - 5)^2 } \, dt \]
        \[ = \int_{0}^{1} \sqrt{2.525 + 4.289} \, dt \]
        \[ = \int_{0}^{1} \sqrt{2.6103} \, dt \]
        \item Por lo tanto la longitud de la valla entre los puntos (5\sqrt{3},5) y (5\sqrt{2},5\sqrt{2})
        \item es de 2.6103.
    \end{itemize}

    
    \subsection*{Pregunta 3}
Evaluar las integrales que se indican:
\begin{enumerate}
\item $\int_{C} F(x,y) , dr$ donde $F(x,y) = \left(\frac{1}{x^2 + 1}\right)$ sobre $r(t) = t\mathbf{i} + t^2\mathbf{j} + t^4\mathbf{k}$ con $t \in [0,1]$.
\begin{enumerate}
\item Calculamos las derivadas:
[
\frac{dx}{dt} = 1, \quad \frac{dy}{dt} = 2t, \quad \frac{dz}{dt} = 4t^{3}
]
\item Sabemos que $F(x,y) = \frac{1}{t^{2}+1}$.
\item Planteamos la integral:
[
\int_{0}^{1} \frac{1}{t^{2}+1} \sqrt{1^{2}\i+ 4t^{2}\j + 16t^{6}\k} , dt
]
[
\int_{0}^{1} \frac{1}{t^{2}+1} \sqrt{4t^{2}} , dt =  \int_{0}^{1} \frac{2t}{t^{2}+1} , dt
]
\item Hacemos cambio de variable $u = t^{2} + 1, , du = 2t , dt$:
[
\int_{0}^{1} \frac{du}{u} = \ln u \Big|_{0}^{1} = \ln(1+1) = \ln(2) = 0.693
]
\item Por lo tanto, la integral es 0.693.
\end{enumerate}

\item $\int_{C} g(x,y) , dx$ donde $g(x, y) = x^2 - y$ sobre la curva $C: x^2 + y^2 = 4$ de $(0, 2)$ a $(\sqrt{2}, \sqrt{2})$.
\begin{enumerate}
\item Parametrizamos la curva:
[
x = 2 \cos t, \quad y = 2 \sin t
]
con $t$ variando de $\frac{\pi}{2}$ a $\frac{\pi}{4}$.
\item Calculamos la derivada de $x$ respecto a $t$:
[
\frac{dx}{dt} = -2 \sin t
]
\item Escribimos la integral de línea:
[
\int_{\frac{\pi}{2}}^{\frac{\pi}{4}} \left((2\cos t)^2 - 2\sin t\right) (-2\sin t) , dt
]
\item Simplificando, obtenemos:
[
\int_{\frac{\pi}{2}}^{\frac{\pi}{4}} 4\sin t , dt
]
\item Invirtiendo los límites de integración:
[
-\int_{\frac{\pi}{4}}^{\frac{\pi}{2}} 4\sin t , dt = -4 \left[-\cos t \right]_{\frac{\pi}{4}}^{\frac{\pi}{2}} = -2\sqrt{2}
]
\item Por lo tanto, la integral de línea es $-2\sqrt{2}$.
\end{enumerate}

\item $\int_{C} \mathbf{F} \cdot d\mathbf{r}$ donde $\mathbf{F}(x, y, z) = z\mathbf{i} + x\mathbf{j} + y\mathbf{k}$ y $C$ es la curva $\mathbf{r}(t) = \sin t \mathbf{i} + \cos t \mathbf{j} + t \mathbf{k}$ con $t \in [0, 2\pi]$.
\begin{enumerate}
\item Calculamos la parametrización y la derivada de la curva:
[
\mathbf{r}(t) = \sin t \mathbf{i} + \cos t \mathbf{j} + t \mathbf{k}, \quad \mathbf{r}'(t) = \cos t \mathbf{i} - \sin t \mathbf{j} + \mathbf{k}
]
\item Evaluamos el campo vectorial sobre la curva:
[
\mathbf{F}(\mathbf{r}(t)) = t \mathbf{i} + \sin t \mathbf{j} + \cos t \mathbf{k}
]
\item Calculamos el producto punto $\mathbf{F}(\mathbf{r}(t)) \cdot \mathbf{r}'(t)$:
[
\mathbf{F}(\mathbf{r}(t)) \cdot \mathbf{r}'(t) = t \cos t - \sin^2 t + \cos t
]
\item Integramos a lo largo de la curva:
[
\int_{0}^{2\pi} (t \cos t - \sin^2 t + \cos t) , dt = -\pi + \pi + 0 = 0
]
\item Por lo tanto, la integral de línea es 0.
\end{enumerate}

\item $\int_{C} f(x, y) , dx$ donde $f(x, y) = \frac{x^3}{y}$ sobre la curva $C: y = \frac{x^2}{2}$ con $0 \leq x \leq 2$.
\begin{align*}
\int_{C} f(x, y) , dx &= \int_{0}^{2} \frac{x^3}{\frac{x^2}{2}} , dx \
&= 2 \int_{0}^{2} x^2 , dx \
&= 2 \cdot \frac{2^3}{3} \
&= \frac{16}{3}
\end{align*}

\item $\int_{C} xy , dx + (x + y) , dy$ sobre $y = x^2$ de $(-1, 1)$ a $(2, 4)$.
\begin{align*}
\int_{C} xy , dx + (x + y) , dy &= \int_{-1}^{2} x \cdot x^2 , dx + \int_{1}^{4} (x + x^2) , dy \
&= \int_{-1}^{2} x^3 , dx + \int_{1}^{4} (x + x^2) , dy \
&= \left[ \frac{x^4}{4} \right]{-1}^{2} + \left[ x^2 + \frac{x^3}{3} \right]{1}^{4} \
&= \left( \frac{2^4}{4} - \frac{(-1)^4}{4} \right) + \left( 4^2 + \frac{4^3}{3} - 1^2 - \frac{1^3}{3} \right) \
&= 8 - \frac{1}{4} + 16 + \frac{64}{3} - 1 - \frac{1}{3} \
&= 22 + \frac{53}{12}
\end{align*}

\item $\int_{C} (x + y + z) , ds$ sobre el segmento de recta que va de $(1, 2, 3)$ a $(0, -1, 1)$.
\begin{align*}
\int_{C} (x + y + z) , ds &= \int_{0}^{1} (1 - t + 2 - t + 3 - t) \sqrt{1^2 + (-1-2)^2 + (1-3)^2} , dt \
&= \int_{0}^{1} 6 - 3t \sqrt{6} , dt \
&= 6\sqrt{6} - \frac{3\sqrt{6}}{2} \
&= \frac{27\sqrt{6}}{2}
\end{align*}

\item $\int_{C} (\sin z , dx + \cos z , dy + (xy)^{1/3} , dz)$ con $C: r(t) = \cos^3 t \mathbf{i} + \sin^3 t \mathbf{j} + t \mathbf{k}$ con $t \in [0, \frac{7\pi}{2}]$.
\begin{align*}
\int_{C} (\sin z , dx + \cos z , dy + (xy)^{1/3} , dz) &= \int_{0}^{7\pi/2} (\sin t \cdot (-3\sin^2 t \cos t) + \cos t \cdot 3\sin^2 t \cos t + (t)^{1/3}) , dt \
&= \int_{0}^{7\pi/2} (-3\sin^3 t \cos t + 3\sin^2 t \cos^2 t + t^{1/3}) , dt \
&= -3 \int_{0}^{7\pi/2} \frac{1}{4} \sin 4t , dt + 3 \int_{0}^{7\pi/2} \frac{1}{2} \cos 2t , dt + \frac{7\pi}{2} \cdot \frac{7\pi}{6} \
&= -\frac{3}{4} + \frac{3}{2} + \frac{49\pi^2}{24} \
&= \frac{49\pi^2}{24} - \frac{1}{4}
\end{align*}

\item $\int_{R} (2xy + \sqrt{z})$ a lo largo de la curva $R(t) = \cos t \mathbf{i} + \sin t \mathbf{j} + t \mathbf{k}$, con $0 \leq t \leq \pi$.
\begin{align*}
\int_{R} (2xy + \sqrt{z}) &= \int_{0}^{\pi} (2(\cos t)(\sin t)(t) + \sqrt{t}) , dt \
&= 2 \int_{0}^{\pi} t \sin 2t , dt + \int_{0}^{\pi} \sqrt{t} , dt \
&= 2 \left[ -\frac{t^2}{4} \cos 2t \right]{0}^{\pi} + \left[ \frac{2}{3}t^{3/2} \right]{0}^{\pi} \
&= 2 \left( -\frac{\pi^2}{4} - 0 \right) + \frac{2}{3} \left( \pi^{3/2} - 0 \right) \
&= -\frac{\pi^2}{2} + \frac{2}{3}\pi^{3/2}
\end{align*}
\end{enumerate}
\end{document}

