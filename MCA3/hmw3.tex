\documentclass{article}
\usepackage[utf8]{inputenc}
\usepackage{amsmath}
\usepackage{amsfonts}
\usepackage{amssymb}

\title{Tarea 3 - MCA III}
\author{David Rivera Morales}
\date{}

\begin{document}

\maketitle

\section*{Preguntas}

\subsection*{Pregunta 1}
\begin{itemize}
    \item \textbf{Hallar la longitud total del alambre representado por la curva \( C = C1 \cup C2 \), donde \( C1 \) es parte de la parábola \( y = x^2 \) y \( C2 \) es un segmento de recta.}
    \begin{itemize}
        \item \textbf{Caso C1:} La parametrización es...
        \item \textbf{Caso C2:} La parametrización es \( r(t) = t\mathbf{i} + t\mathbf{j} \) con \( t \in [c, d] \).
    \end{itemize}

    \item \textbf{Si la densidad en cada punto del alambre está dada por la función \( \rho(x, y, z) = x + \sqrt{y} - z^2 \), hallar la masa total del alambre.}
    
    \item \textbf{Hallar la densidad promedio del alambre.}
    \begin{itemize}
        \item La fórmula del valor promedio es \( \frac{\int_{C} \rho(x, y, z) \, ds}{\int_{C} ds} \).
        \item Sabiendo que la longitud del arco calculada es \( \int_{C} ds = 2.478 \), se calcula \( \frac{\int_{C} \rho(x, y, z) \, ds}{2.478} \).
        \item Haciendo el cambio de variables \( dx = dt \), \( dy = y \, dt \), \( dz = 2t \, dt \), entonces \( ds = \sqrt{(dx)^2 + (dy)^2 + (dz)^2} = \sqrt{(1)^2 + (y)^2 + (2t)^2} \, dt \).
        \item La función densidad parametrizada será:
        \[ \rho(t, y(t), z(t)) = t + \sqrt{t} - 4t^2 \]
        \item La integral requerida es:
        \[ \int_{0}^{1} \left( t + \sqrt{t} - 4t^2 \right) \sqrt{1 + t + 4t^2} \, dt \]
    \end{itemize}
\end{itemize}

\subsection*{Pregunta 2}
\begin{itemize}
    \item \textbf{Si con 1 L de pintura se pueden cubrir 100 m\(^2\), ¿cuántos litros de pintura se necesitan para pintar ambos lados de la valla?}
    \begin{itemize}
        \item La base de la valla está descrita por \( x = 10 \cos t \), \( y = 10 \sin t \), según la parametrización \( r(t) = 10 \cos t \mathbf{i} + 10 \sin t \mathbf{j} \).
        \item Entonces \(\frac{dx}{dt} = -10 \sin t \), \(\frac{dy}{dt} = 10 \cos t \), y planteando una integral y sabiendo que \(z(x, y) = 0.01(x^2 - y^2) + 4\), se tiene:
        \item \to\[ \int_{0}^{2\pi} \left(0.1( \cos^2 t - \sin^2 t) + 4\right) \sqrt{(-10 \sin t)^2 + (10 \cos t)^2 + 1} \, dt \]
        \item \to\[ \int_{0}^{2\pi} \left(0.1( \cos^2 t -  \sin^2 t) + 4\right) \sqrt{100} \, dt \]
        \item \to\[ \int_{0}^{2\pi} \left(0.1( \cos^2 t -  \sin^2 t) + 4\right) 10 \, dt \]
        \item \to\[ \int_{0}^{2\pi} \left( \cos^2 t -  \sin^2 t + 40\right) \, dt \]
        \item \to\[ (\frac{\cos t \sin t}{2} - \frac{\sin t \cos t}{2} + 40t) \Big|_{0}^{2\pi} \]
        \item \to\[ 80\pi \]
        
        
    \end{itemize}
    
    \item \textbf{¿Cuál es la altura promedio que tiene la valla?}
    \begin{itemize}
        \item Utilizando el valor de \( \int_{C} P(x, y, z) \, dz \) y los derivados \( x' = -10 \sin t \), \( y' = 10 \cos t \), \( z' = 0 \), se calcula la altura promedio con la fórmula:
        \[ \frac{\int_{C} P(x, y, z) \, dz}{\int_{C} ds} \] \to \[ \frac{\int_{0}^{2\pi} (0.01 (10 \cos^2 t^ - 10 \sin^2 t)+4) \sqrt{(-10 \sin t)^2 + (10 \cos t)^2 + 1} \, dt}{\int_{0}^{2\pi} \sqrt{(-10 \sin t)^2 + (10 \cos t)^2 } \, dt} \] \to         \[ \frac{80\pi m^2}{\int_{0}^{2\pi} \sqrt{100(\cos^2 t + \sin^2 t)} \, dt} \]   \end{itemize}
        \[ \frac{80\pi m^2}{\int_{0}^{2\pi} 10 \, dt m} \] \to \[ \frac{80\pi m^2}{20\pi m} \] \to \[ 4 m \]
        \item Por lo tanto, la altura promedio de la valla es de 4 metros.
    
    \end{itemize}
\subsection*{Pregunta 3}
Evaluar las integrales que se indican:
\begin{enumerate}
    \item \( \int_{C} F(x,y) \, dr \) donde \( F(x,y) = \left(\frac{1}{x^2 + 1}\right) \) sobre \( r(t) = t\mathbf{i} + t^2\mathbf{j} + t^4\mathbf{k} \) con \( t \in [0,1] \).
    \item \( \int_{C} g(x,y) \, dx \) donde \( g(x, y) = x^2 - y \) sobre la curva \( C: x^2 + y^2 = 4 \) de \( (0, 2) \) a \( (\sqrt{2}, \sqrt{2}) \).
    \item \( \int_{C} F \cdot dr \) donde \( F(x, y, z) = z\mathbf{i} + x\mathbf{j} + y\mathbf{k} \) y \( C \) es la curva \( r(t) = \sin t \mathbf{i} + \cos t \mathbf{j} + t \mathbf{k} \) con \( t \in [0, 2\pi] \).
    \item \( \int_{C} f(x, y) \, dx \) donde \( f(x, y) = \frac{x^3}{y} \) sobre la curva \( C: y = \frac{x^2}{2} \) con \( 0 \leq x \leq 2 \).
    \item \( \int_{C} xy \, dx + (x + y) \, dy \) sobre \( y = x^2 \) de \( (-1, 1) \) a \( (2, 4) \).
    \item \( \int_{C} (x + y + z) \, ds \) sobre el segmento de recta que va de \( (1, 2, 3) \) a \( (0, -1, 1) \).
    \item \( \int_{C} (\sin z \, dx + \cos z \, dy + (xy)^{1/3} \, dz) \) con \( C: r(t) = \cos^3 t \mathbf{i} + \sin^3 t \mathbf{j} + t \mathbf{k} \) con \( t \in [0, \frac{7\pi}{2}] \).
    \item \( \int_{R} (2xy + \sqrt{z}) \) a lo largo de la curva \( R(t) = \cos t \mathbf{i} + \sin t \mathbf{j} + t \mathbf{k} \), con \( 0 \leq t \leq \pi \).
\end{enumerate}
\end{document}