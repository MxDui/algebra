\documentclass{article}
\usepackage[utf8]{inputenc}
\usepackage{amsmath}
\usepackage{amsfonts}
\usepackage{amssymb}

\begin{document}

\title{Tarea de Matemáticas}
\author{Tu Nombre}
\date{\today}
\maketitle

\section*{Preguntas}

\subsection*{Pregunta 1}
\begin{enumerate}
    \item Hallar la longitud total del alambre representado por la curva \( C = C1 \cup C2 \), donde \( C1 \) es parte de la parábola \( y = x^2 \) y \( C2 \) es un segmento de recta.
    \item La longitud total del alambre para \( C_1 \) es:
    \[
    L_{C_1} = \int_{0}^{1} \sqrt{1 + (2t)^2} \, dt = \frac{\text{asinh}(2)}{4} + \frac{\sqrt{5}}{2}
    \]
    Y para \( C_2 \) es \( 1 \). Por lo tanto, la longitud total es \( L_{C_1} + 1 \).
    
    \item La masa total del alambre para \( C_1 \) es:
    \[
    m_{C_1} = \int_{0}^{1} (t + \sqrt{t^2}) \sqrt{1 + (2t)^2} \, dt = \frac{5\sqrt{5}}{6} - \frac{1}{6}
    \]
    Y para \( C_2 \) es \( 1 \). Entonces, la masa total es \( m_{C_1} + 1 \).
    
    \item La densidad promedio del alambre se calculará como la masa total dividida por la longitud total. Añadiremos este cálculo a continuación:
    \[
    \rho_{\text{promedio}} = \frac{m_{C_1} + 1}{L_{C_1} + 1} = \frac{\frac{5\sqrt{5}}{6} - \frac{1}{6} + 1}{\frac{\text{asinh}(2)}{4} + \frac{\sqrt{5}}{2} + 1}
    \]
    \item Si la densidad en cada punto del alambre está dada por la función \( \rho(x, y, z) = x + \sqrt{y} - z^2 \), hallar la masa total del alambre.
    \item Hallar la densidad promedio del alambre.
\end{enumerate}

\subsection*{Pregunta 2}
\begin{enumerate}
    \item Si con 1 L de pintura se pueden cubrir 100 m\(^2\), ¿cuántos litros de pintura se necesitan para pintar ambos lados de la valla?
    \item ¿Cuál es la altura promedio que tiene la valla?
    \item ¿Cuál es la longitud de la valla entre los puntos \( (5\sqrt{3}, 5) \) y \( (5\sqrt{2}, 5\sqrt{2}) \)?
\end{enumerate}

\subsection*{Pregunta 3}
Evaluar las integrales que se indican:
\begin{enumerate}
    \item \( \int_{C} F(x,y) \, dr \) donde \( F(x,y) = \left(\frac{1}{x^2 + 1}\right) \) sobre \( r(t) = t\mathbf{i} + t^2\mathbf{j} + t^4\mathbf{k} \) con \( t \in [0,1] \).
    \item \( \int_{C} g(x,y) \, dx \) donde \( g(x, y) = x^2 - y \) sobre la curva \( C: x^2 + y^2 = 4 \) de \( (0, 2) \) a \( (\sqrt{2}, \sqrt{2}) \).
    \item \( \int_{C} F \cdot dr \) donde \( F(x, y, z) = z\mathbf{i} + x\mathbf{j} + y\mathbf{k} \) y \( C \) es la curva \( r(t) = \sin t \mathbf{i} + \cos t \mathbf{j} + t \mathbf{k} \) con \( t \in [0, 2\pi] \).
    \item \( \int_{C} f(x, y) \, dx \) donde \( f(x, y) = \frac{x^3}{y} \) sobre la curva \( C: y = \frac{x^2}{2} \) con \( 0 \leq x \leq 2 \).
    \item \( \int_{C} xy \, dx + (x + y) \, dy \) sobre \( y = x^2 \) de \( (-1, 1) \) a \( (2, 4) \).
    \item \( \int_{C} (x + y + z) \, ds \) sobre el segmento de recta que va de \( (1, 2, 3) \) a \( (0, -1, 1) \).
    \item \( \int_{C} (\sin z \, dx + \cos z \, dy + (xy)^{1/3} \, dz) \) con \( C: r(t) = \cos^3 t \mathbf{i} + \sin^3 t \mathbf{j} + t \mathbf{k} \) con \( t \in [0, \frac{7\pi}{2}] \).
    \item \( \int_{R} (2xy + \sqrt{z}) \) a lo largo de la curva \( R(t) = \cos t \mathbf{i} + \sin t \mathbf{j} + t \mathbf{k} \), con \( 0 \leq t \leq \pi \).
\end{enumerate}
\end{document}


\end{document}
