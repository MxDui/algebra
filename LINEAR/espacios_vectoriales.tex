\subsection*{Definiciones}
Sea \( (k, +_k, \cdot_k) \) un campo. Diremos que \( (V, +, \cdot) \) es un \( k \)-espacio vectorial si \( V \) es un conjunto con dos operaciones (funciones de conjuntos):
Una suma \( + : V \times V \rightarrow V \) y un producto por escalares \( \cdot : k \times V \rightarrow V \) que satisfacen las siguientes propiedades.

Denotemos para todo \( \alpha, \gamma \in V \) y \( \alpha \in k \), \( +(\alpha, \gamma) =: \alpha + \gamma \) y \( \cdot(\alpha, x) =: \alpha x \).

Para todos \( x, y, z \in V \) y para todos \( \alpha, \beta \in k \).
\begin{itemize}
    \item[+1.] Conmutatividad. \( x + y = y + x \)
    \item[+2.] Asociatividad. \( x + (y + z) = (x + y) + z \)
    \item[+3.] Neutro aditivo. Existe \( 0 \in V \) tal que \( x + 0 = x = 0 + x \)
    \item[+4.] Inversos aditivos. Para todo \( x \in V \), existe \( y \in V \) tal que \( x + y = 0 = y + x \)
\end{itemize}

\begin{itemize}
    \item[.1.] Asociatividad. \( (\alpha \cdot \beta)x = \alpha(\beta x) \).
    \item[.2.] Neutro multiplicativo. \( 1_k x = x \)
\end{itemize}

Distributividades:
\begin{itemize}
    \item[d.1.] \( (\alpha + \beta)x = (\alpha x) + (\beta x) \)
    \item[d.2.] \( \alpha(x + y) = (\alpha x) + (\alpha y) \).
\end{itemize}

En este caso, a los elementos del campo \( k \), los llamaremos \textit{escalares} y a los elementos de \( V \) los llamaremos \textit{vectores}.

Nota: Las propiedades anteriores se llaman los Axiomas de los Espacios Vectoriales. Es decir, son las reglas del juego para jugar a los Espacios Vectoriales.

\subsection*{Ejemplos}

\subsection*{Problemas}